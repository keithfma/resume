%!TEX TS-program = xelatex
%!TEX encoding = UTF-8 Unicode
\documentclass[11pt, letter]{awesome-cv}
\fontdir[fonts/] % override a directory location for fonts(default: 'fonts/')
\newcommand*{\sectiondir}{resume/} % Configure a directory location for sections

%%% Override color
% Awesome Colors: awesome-emerald, awesome-skyblue, awesome-red, awesome-pink, awesome-orange
%                 awesome-nephritis, awesome-concrete, awesome-darknight
%% Color for highlight
% Define your custom color if you don't like awesome colors
\colorlet{awesome}{awesome-red}
%\definecolor{awesome}{HTML}{CA63A8}
%% Colors for text
%\definecolor{darktext}{HTML}{414141}
%\definecolor{text}{HTML}{414141}
%\definecolor{graytext}{HTML}{414141}
%\definecolor{lighttext}{HTML}{414141}

%%% Override a separator for social informations in header(default: ' | ')
%\headersocialsep[\quad\textbar\quad]

\usepackage{import} % Needed to divide into several files


% Header --------------------------------------------------------------------------------------------

\begin{document}

\begin{center}
\headerlastnamestyle{Keith F. Ma}
\vspace{0.4mm}

\headeraddressstyle{keithfma@gmail.com}
\quad\textbar\quad
\headeraddressstyle{610-389-1406}
\quad\textbar\quad
\headeraddressstyle{Cambridge MA}
\quad\textbar\quad
\headeraddressstyle{\href{http://github.com/keithfma}{github.com/keithfma}}
\quad\textbar\quad
\headeraddressstyle{\href{http://linkedin.com/in/keithfma}{linkedin.com/in/keithfma}}


\end{center}
  
% body -------------------------------------------------------------------------------------------------

\cvsection{Skills}
\begin{cvskills}
\cvskill{Languages}{Python, MATLAB, C, C++, Fortran, Javascript, SQL, Bash}
\cvskill{Tools}{scipy, sklearn, flask, OpenMP/MPI, Elasticsearch, PostgreSQL, MongoDB, GDAL, Sun Grid Engine}
\end{cvskills}

\cvsection{Experience}
\begin{cventries}
  \cventry
    {Data Science Fellow}
    {Insight Data Science}
    {Boston, MA}
    {June 2018 - Present}
    {
    \begin{cvitems}
      \item Built a navigation app that optimizes for comfortable outdoor travel by simulating sunlight along potential routes from dense LiDAR points and the OpenStreetMap graph (\emph{Parasol}, http://parasol.allnans.com).
      \item Enabled users to balance the trade-off between route length and sun/shade preference with an adjustable cost function using a single free parameter. 
      \item Implemented scalable data processing pipeline to fit surfaces to billions of LiDAR observations and integrate solar input along thousands of road segments for any date/time.
    \end{cvitems}
    }
  \cventry
    {Research Scientist}
    {Boston Fusion Corp}
    {Lexington, MA}
    {June 2016 - May 2018}
    {
      \begin{cvitems}
        \item {Designed and implemented practical analytical tools that leverage machine learning to address a wide range of Department of Defense needs for DARPA, ONR, etc. Projects included adaptive decision support systems, data exploration via multi-level graphs, gaming environment artificial intelligence, and statistical models for missile detection.}
        \item {Modernized internal software engineering practices, including wrapping new capabilities as services with REST APIs, standardized packaging, sensible unit testing, and automatic documentation generation.}
        \item {Presented scientific results and software demonstrations to diverse commercial and military clients and stakeholders, and managed partnerships with commercial and academic collaborators.}
      \end{cvitems}
    }
  \cventry
    {Scientific Programmer / Analyst}
    {Boston University Research Computing Services}
    {Boston, MA}
    {Sept. 2014 - June 2016}
    {
      \begin{cvitems}
        \item {Consulted with BU faculty and students to design, optimize, and parallelize research software and computational workflows for HPC clusters. For example, improved throughput for a large-scale satellite imagery analysis (\textasciitilde70 TB) by reducing I/O bottlenecks and distributing independent tasks to many nodes.}
        \item {Promoted best practices in scientific computing by organizing and teaching tutorials and workshops. Led the expansion of offerings through new partnerships with Software Carpentry and XSEDE.}
      \end{cvitems}
    }
  \cventry
    {Doctoral Candidate}
    {Yale University Department of Geology \& Geophysics}
    {New Haven, CT}
    {Sept. 2007 - Present}
    {
      \begin{cvitems}
        \item {Designed and built a distributed numerical model to simulate erosion by rivers and glaciers that significantly improved the accuracy, stability, functionality, and speed of earlier models.}
        \item {Created a method for reconstructing past topography by decomposing and scaling features created by tectonics and erosion and applied to infer topographic history of the Patagonian Andes.}
        \item{Built an analog "sandbox" model of mountain formation and developed custom software for instrument control and automated image processing to measure experimental velocity fields.}
      \end{cvitems} 
    }
\end{cventries}

\cvsection{Education}
\begin{cventries}
  \cventry
    {PhD in Geology \& Geophysics}
    {Yale University}
    {New Haven, CT}
    {Expected Fall 2018}
    {}
  \cventry
    {BA in Geology - Biology}
    {Brown University}
    {Providence RI}
    {May, 2005}
    {}
\end{cventries}

\end{document}
